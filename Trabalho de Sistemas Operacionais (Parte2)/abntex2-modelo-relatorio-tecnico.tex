%% abtex2-modelo-relatorio-tecnico.tex, v-1.7.1 laurocesar
%% Copyright 2012-2013 by abnTeX2 group at http://abntex2.googlecode.com/ 
%%
%% This work may be distributed and/or modified under the
%% conditions of the LaTeX Project Public License, either version 1.3
%% of this license or (at your option) any later version.
%% The latest version of this license is in
%%   http://www.latex-project.org/lppl.txt
%% and version 1.3 or later is part of all distributions of LaTeX
%% version 2005/12/01 or later.
%%
%% This work has the LPPL maintenance status `maintained'.
%% 
%% The Current Maintainer of this work is the abnTeX2 team, led
%% by Lauro César Araujo. Further information are available on 
%% http://abntex2.googlecode.com/
%%
%% This work consists of the files abntex2-modelo-relatorio-tecnico.tex,
%% abntex2-modelo-include-comandos and abntex2-modelo-references.bib
%%

% ------------------------------------------------------------------------
% ------------------------------------------------------------------------
% abnTeX2: Modelo de Relatório Técnico/Acadêmico em conformidade com 
% ABNT NBR 10719:2011 Informação e documentação - Relatório técnico e/ou
% científico - Apresentação
% ------------------------------------------------------------------------ 
% ------------------------------------------------------------------------

% Alterado por Rodrigo Campiolo para apresentação de relatórios na disciplina
% de Redes de Computadores II do Bacharelado em Ciência da Computação da UTFPR-CM.


\documentclass[
	% -- opções da classe memoir --
	12pt,				% tamanho da fonte
	%openright,			% capítulos começam em pág ímpar (insere página vazia caso preciso)
	oneside,   	        % para impressão em verso e anverso use twoside. Oposto a oneside
	a4paper,			% tamanho do papel. 
	% -- opções da classe abntex2 --
	%chapter=TITLE,		% títulos de capítulos convertidos em letras maiúsculas
	%section=TITLE,		% títulos de seções convertidos em letras maiúsculas
	%subsection=TITLE,	% títulos de subseções convertidos em letras maiúsculas
	%subsubsection=TITLE,% títulos de subsubseções convertidos em letras maiúsculas
	% -- opções do pacote babel --
	english,			% idioma adicional para hifenização
	french,				% idioma adicional para hifenização
	spanish,			% idioma adicional para hifenização
	brazil,				% o último idioma é o principal do documento
	]{pacotes/abntex2}


% ---
% PACOTES
% ---

% ---
% Pacotes fundamentais 
% ---
\usepackage{cmap}				% Mapear caracteres especiais no PDF
\usepackage{lmodern}			% Usa a fonte Latin Modern
\usepackage[T1]{fontenc}		% Selecao de codigos de fonte.
\usepackage[utf8]{inputenc}		% Codificacao do documento (conversão automática dos acentos)
\usepackage{indentfirst}		% Indenta o primeiro parágrafo de cada seção.
\usepackage{color}				% Controle das cores
\usepackage{graphicx}			% Inclusão de gráficos
% ---
\usepackage[utf8]{inputenc}

\usepackage{float}
% ---
% Pacotes adicionais, usados no anexo do modelo de folha de identificação
% ---
\usepackage{multicol}
\usepackage{multirow}
% ---
	
% ---
% Pacotes adicionais, usados apenas no âmbito do Modelo Canônico do abnteX2
% ---
\usepackage{lipsum}				% para geração de dummy text
% ---

% ---
% Pacotes de citações
% ---
\usepackage[brazilian,hyperpageref]{backref}	 % Paginas com as citações na bibl
\usepackage[alf]{pacotes/abntex2cite}	% Citações padrão ABNT
\usepackage{comment}
% --- 
% CONFIGURAÇÕES DE PACOTES
% --- 

% % ---
% % Configurações do pacote backref
% % Usado sem a opção hyperpageref de backref
% \renewcommand{\backrefpagesname}{Citado na(s) página(s):~}
% % Texto padrão antes do número das páginas
% \renewcommand{\backref}{}
% % Define os textos da citação
% \renewcommand*{\backrefalt}[4]{
% 	\ifcase #1 %
% 		Nenhuma citação no texto.%
% 	\or
% 		Citado na página #2.%
% 	\else
% 		Citado #1 vezes nas páginas #2.%
% 	\fi}%
% ---

% ---
% Informações de dados para CAPA e FOLHA DE ROSTO
% ---
\titulo{Resumo dos Capítulos 20 e 21 do Livro de Sistemas Operacionais 3º Edição (Parte2)}
\autor{Igor Ortega Carmona | RA: 00236524 \\ Jonathan Fernando Thomaz Melo | RA: 00229289}
\local{Cianorte}
\data{Novembro / 2022}
\instituicao{%
  UNIPAR - CIANORTE
  \par
  Estudantes de Tecnologia em Análise e Desenvolvimento de Sistemas (ADS)
}
\tipotrabalho{Relatório técnico}
% O preambulo deve conter o tipo do trabalho, o objetivo, 
% o nome da instituição e a área de concentração 
\preambulo{Resumo dos capítulos 20 e 21 do livro de Sistemas Operacionais 3º Edição, onde os tópicos a serem resumidos foram orientados pelo professor Dr. Roni da UNIPAR pela matéria de \textit{Sistemas Operacionais}, grade do curso de \textit{Análise e Desenvolvimento de Sistemas}. }
% ---

% ---
% Configurações de aparência do PDF final

% alterando o aspecto da cor azul
\definecolor{blue}{RGB}{41,5,195}

% informações do PDF
\makeatletter
\hypersetup{
     	%pagebackref=true,
		pdftitle={\@title}, 
		pdfauthor={\@author},
    	pdfsubject={\imprimirpreambulo},
	    pdfcreator={LaTeX with abnTeX2},
		pdfkeywords={abnt}{latex}{abntex}{abntex2}{relatório técnico}, 
		colorlinks=true,       		% false: boxed links; true: colored links
    	linkcolor=blue,          	% color of internal links
    	citecolor=blue,        		% color of links to bibliography
    	filecolor=magenta,      		% color of file links
		urlcolor=blue,
		bookmarksdepth=4
}
\makeatother
% --- 

% --- 
% Espaçamentos entre linhas e parágrafos 
% --- 

% O tamanho do parágrafo é dado por:
\setlength{\parindent}{1.3cm}

% Controle do espaçamento entre um parágrafo e outro:
\setlength{\parskip}{0.2cm}  % tente também \onelineskip

% ---
% compila o indice
% ---
\makeindex
% ---

% Omite a numeração de capítulos
\renewcommand*\thesection{\arabic{section}}



% ----
% Início do documento
% ----
\begin{document}

% Retira espaço extra obsoleto entre as frases.
\frenchspacing 

% ----------------------------------------------------------
% ELEMENTOS PRÉ-TEXTUAIS
% ----------------------------------------------------------
% \pretextual

% ---
% Capa
% ---
%\imprimircapa
% ---

% ---
% Folha de rosto
% (o * indica que haverá a ficha bibliográfica)
% ---
\imprimirfolhaderosto
% ---


% ---
% RESUMO
% ---

% resumo na língua vernácula (obrigatório)
%\begin{resumo}
 
 % Foram redigidos conceitos de recursividade explicando desde a sua teoria base até a implementação final sempre visando e relembrando dos cuidados a se tomar com a demonstração de exemplos na prática e detalhando o \textit{"mind-set"} dos passos realizados para melhor compreendimento.

 % \vspace{\onelineskip}
    
 % \noindent
 % \textbf{Palavras-chave}: Recursão. Árvore binária, linguagem C.
%\end{resumo}
% ---

% ---
% inserir lista de ilustrações
% ---
%\pdfbookmark[0]{\listfigurename}{lof}
%\listoffigures*
%\cleardoublepage
% ---

% ---
% inserir lista de tabelas
% ---
%\pdfbookmark[0]{\listtablename}{lot}
%\listoftables*
%\cleardoublepage
% ---

% ---
% inserir lista de abreviaturas e siglas
% ---
%\begin{siglas}
%  \item[IP] Internet Protocol
%  \item[TCP] Transmission Control Protocol
%  \item[UDP] User Datagram Protocol
%\end{siglas}
% ---

% ---
% inserir o sumario
% ---
\pdfbookmark[0]{\contentsname}{toc}
\tableofcontents*
\cleardoublepage
% ---

% ----------------------------------------------------------
% ELEMENTOS TEXTUAIS
% ----------------------------------------------------------

%------------------------------------------------------------------------%
%% PARA FAZER CITAÇÕES
  %  \cite{kernel/Linux}.
%------------------------------------------------------------------------%


%------------------------------------------------------------------------%
%%LISTAGEM DE ITENS

%\begin{itemize}
%    \item \textbf{Virtual Box 6.1:} TEXTO
%\end{itemize}
%\begin{itemize}
%    \item \textbf{Distribuição Linux GNU/Debian 11.4:} TEXTO
%\end{itemize}
%\begin{itemize}
%    \item \textbf{kernel Linux 5.19.2:} TEXTO
%\end{itemize}
%------------------------------------------------------------------------%


%-------------------------------------------------------------------------%
  %% COLOCAR SITES
  
 % \url{https://www.debian.org/download}.
%-------------------------------------------------------------------------%
 
 
%-------------------------------------------------------------------------%    
%%%% COLOCAR FIGURAS %%%%

 %   \begin{figure}[H]
  %\centering
  %\includegraphics[scale=0.8]{figuras/vm.png}
  %\caption{Configurações inicias da máquina}
  %\label{fig:partições}
%\end{figure}
%------------------------------------------------------------------------%

\textual

\makeatletter
\renewcommand{\chapter}{\@gobbletwo}
\makeatother

\section{\textbf{Como o Linux gerencia memória}}
\label{sec:arq-linux}

\subsection{\textbf{Organização da memória}}
\label{subsec:organizacao-memoria}

O gerenciador de memória suporta endereços de 32bits e 64bits. A memória física é dividida em molduras de página de tamanhos fixos, podendo em cada arquitetura em individual ter tamanhos maiores (4 MB) ou menores (4 KB ou 8 KB), o responsável por armazenar informações sobre cada moldura de página é o núcleo,

O sistema de gerenciamento de memória divide o espaço de endereçamento físico em três zonas na qual o tamanho de cada uma delas vai depender da arquitetura. Sendo eles:

\begin{itemize}
    \item \textbf{Primeira zona:} \textit{Memória DMA}.
    \item \textbf{Segunda zona:} \textit{Memória normal}.
    \item \textbf{Terceira zona:} \textit{Memória alta}.
\end{itemize}

Faltas de página podem reduzir o desempenho do núcleo.

\subsection{\textbf{Organização da memória virtual}}
\label{subsec:organizacao-memoria-virtual}

O sistema de memória virtual suporta até três níveis de tabelas de páginas, serve para realizar a localização dos mapeamentos entre páginas virtuais e molduras de páginas. Sendo a sua hierarquia:

\begin{itemize}
    \item \textbf{Primeiro nível:} \textit{diretório global de páginas}.
    \item \textbf{Segundo nível:} \textit{diretórios intermediários de páginas}.
    \item \textbf{Terceiro nível:} \textit{tabelas de páginas}.
\end{itemize}

O uso de memória virtual tem como benefícios: um processo pode executar sem ter todas as instruções e dados dentro da memória principal. A memória virtual é criada de forma automática quando instalamos um S.O.

\section{\textbf{Como o Linux gerencia arquivos}}
\label{sec:arq-arquivos}

Os arquivos servem como pontos de acesso a dados na qual eles podem ser encontrados em um disco local. O Linux suporta diversos sistemas de arquivos. O sistema de arquivos possui níveis de diretórios e são organizados em uma árvore na qual possui caminhos.

\section{\textbf{Como o Windows XP gerencia a memória}}
\label{sec:arq-memoria-windows}

O gerenciador de memória virtual (VMM) do Windows cria a ilusão de que cada
processo tem um espaço de memória próxima de 4 GB. O VMM armazena alguns
dados em disco em arquivos denominados arquivos de páginas, já que o sistema
aloca mais memória virtual a processos do que a memória principal pode comportar.
Windows XP divide memória virtual em molduras de páginas na memória principal ou
em arquivos em disco. Tendo um tamanho fixo e um sistema hierárquico de
endereçamento de dois níveis. Como estratégia, o VMM usa páginas de
cópia-na-escrita e emprega alocação tardia, adiando até que seja absolutamente
necessário. Portanto, quando o VMM é forçado a executar E/S de disco, ele procura
páginas no disco e coloca essas páginas na memória principal antes que seja
necessário. A heurística assegura que o ganho na taxa de E/S de disco compensa o
custo de lotar a memória principal com páginas potencialmente não utilizadas.
Quando a RAM fica cheia, o Windows realiza uma versão do algoritmo de
substituição de página menos usada.


\section{\textbf{Como o Windows XP gerencia os arquivos}}
\label{sec:arq-arquivos-windows}

Os sistemas de arquivos do Windows XP, neste caso, o NTFS, consiste em três
camadas de drivers. Na camada mais inferior contém vários drivers de volume que
controlam um dispositivo de hardware específico, como o disco rígido. Drivers de
sistema de arquivo, que compõem o próximo nível, implementam um formato
particular de sistema de arquivos, esses drivers implantam o que um usuário típico
vê como um sistema de arquivo: uma organização hierárquica de arquivos e as
funções relacionadas que os manipulam. Por fim, drivers de filtros de sistemas de
arquivos executam tarefas de alto nível, como proteção contra vírus, compressão e
criptografia.

\newpage
% ----------------------------------------------------------
% ELEMENTOS PÓS-TEXTUAIS
% ----------------------------------------------------------
\postextual
% ----------------------------------------------------------
% Referências bibliográficas
% ----------------------------------------------------------
\renewcommand{\bibsection}{%
\section{\bibname}
\bibmark
%\ifnobibintoc\else
%\phantomsection
%\addcontentsline{toc}{section}{\bibname}
%\fi
\prebibhook}

\bibliography{abntex2-modelo-references}

% ----------------------------------------------------------
% Apêndices
% ----------------------------------------------------------

% ---
% Inicia os apêndices
% ---
% \begin{apendicesenv}

% % ----------------------------------------------------------
% \section*{Apêndice A - Nome do Apêndice}
% \addcontentsline{toc}{section}{Apêndice A - Nome do Apêndice}
% % ----------------------------------------------------------

% \end{apendicesenv}
% % ---


% ----------------------------------------------------------
% Anexos
% ----------------------------------------------------------

% % ---
% % Inicia os anexos
% % ---
% \begin{anexosenv}

% % ---
% \section*{Anexo A - Nome do Anexo}
% \addcontentsline{toc}{section}{Anexo A - Nome do Anexo}
% % ---
% \end{anexosenv}


\end{document}
